\chapter{Conclusioni e sviluppi futuri}\label{conclusioni}
E' stato sviluppato un software P2P per lo scambio di messaggi tra due o più utenti facendo uso del
linguaggio C e dell'API del socket di Berkeley.
L'API del socket di Berkeley è stata utilizzata per la gestire la connessione tra processi remoti, permettendo 
l'instaurazione di connessioni di tipo persistente; inoltre è stato inserito un livello software ex-novo, il quale si comporta come un'interfaccia d'accesso per la comunicazione tra lo strato TCP e quello HTTP su cui si basa l'intero processo di scmabio di pacchetti.
I vari test che sono stati eseguiti hanno dimostrato che il sistema riesce a mantenere un buon grado di stabilità e velocità, senza mai trovarsi in uno stato di inconsisteza.
Nonostante ciò effettuando un'analisi più dettagliata si possono notare alcune inefficenze, soprattuto per quello che riguarda la gestione dell'invio conocorrente dei pacchetti. 
A tale proposito è stato deciso di adottare un meccanismo di sequenzializzazione dei pacchetti basato sull'utilizzo di particolari strutture come le code e su un algoritmo di polling basato sullo switching del thread client nello stato di sleep.
In questo modo ogni client non rischierà di perdere alcun segnale di notifica ,proveniente dal lato server,per l'invio di un pacchetto.
Una possibile risoluzione al fine di evitare lo sleeping del thread potrebbe essere l'introduzione di un pthread_cond_timedwait(), ovvero una funzione che gestisce la sospensione del thread solo nel caso in cui la condizione posta sia verificata.
Per quello che rigurada lo sviluppo di funzionalità aggiuntive si potrebbe pensare alla possibilità di effettuare il trasferimento di file tra gli utenti di una chat.   
