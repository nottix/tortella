\section{Init}
Il modulo in questione consente di effettuare il bootstrap dell'applicazione, poichè permette ad un peer di conoscere i propri vicini (appositamente memorizzati in un file). Il file dei peer vicini è strutturato nel seguente modo:
\begin{lstlisting}
ip;porta;
ip;porta;
...
\end{lstlisting}
Per la memorizzazione dei record presenti all'interno del file è stata utilizzata la seguente struttura dati:
\begin{lstlisting}
struct init_data {
	char *ip;
	u_int4 port;
};
typedef struct init_data init_data;
\end{lstlisting}
e per la gestione di questi sono state utilizzate le seguenti funzioni.
\paragraph{init\_char\_to\_initdata()}
Riceve come parametro un buffer contenente un record del file e istanzia la struttura dati settando in modo opportuno ip e porta.
\begin{lstlisting}
init_data *init_char_to_initdata(char *buffer)
\end{lstlisting}
\paragraph{init\_read\_file()}
Legge il file in cui sono contenuti i record dei "vicini" e aggiunge tutti i peer presenti all'interno del file in una lista contenente strutture di tipo init_data. Si serve della funzione \texttt{init\_char\_to\_init\_data()} per l'istanziaziazione delle strutture dati.
\begin{lstlisting}
GList *init_read_file(const char *filename)
\end{lstlisting}


