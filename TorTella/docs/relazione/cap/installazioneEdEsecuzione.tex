\chapter{Installazione ed Esecuzione}
Nel presente capitolo verrà fornita una rapida guida d'installazione per l'utente e verrà illustrato un esempio di funzionamento dell'applicazione.
\section{Installazione}
Per generare il Makefile bisogna seguire i seguenti passi: 
\begin{enumerate}
\item Posizionarsi nella radice del progetto; 
\item Eseguire il file ./autogen.sh;
\begin{figure}[H]
\begin{center}
\includegraphics[scale=0.5]{etc/autogen}
\caption{autogen}
\label{autogen}
\end{center}
\end{figure}
\item Eseguire il ./configure sempre nella radice del progetto;
\begin{figure}[H]
\begin{center}
\includegraphics[scale=0.5]{etc/configure}
\caption{configure}
\label{configure}
\end{center}
\end{figure}
\item Ora il Makefile è pronto, per compilare basta eseguire make dalla radice del progetto.
\begin{figure}[H]
\begin{center}
\includegraphics[scale=0.5]{etc/make}
\caption{make}
\label{make}
\end{center}
\end{figure}
\end{enumerate}
\section{Avvio dell'applicazione}
Prima di avviare il programma è necessario che i file vengano creati correttamente. A tal proposito si presenta un esempio di come dev'essere strutturato il file di configurazione e un esempio di come dev'essere strutturato il file contenente i "vicini" conosciuti:
\begin{lstlisting}
#TorTella Configuration file

#LOGGER
verbose = 3;

#SOCKET
qlen = 5;
buffer_len = 4096;

#PACKET
path = /tmp/tortella1;

#UTILS
gen_start = 100000;

#SERVENT
max_len = 4000;
max_thread = 20;
max_fd = 100;
timer_interval = 20;
new_connection_counter = 10000;

#SUPERNODE
datadir = ./data;

#SERVENT
local_ip = 127.0.0.1;
local_port = 2110;
nick = simo;
\end{lstlisting}
Si consiglia all'utente inesperto di cambiare solamente i parametri relativi all'ip, alla porta locale e al nickname da utilizzare.
Per quanto concerne il file di inizializzazione, questo dovrebbe essere strutturato in questo modo:
\begin{lstlisting}
127.0.0.1;2110;
127.0.0.1;2120;
\end{lstlisting}
Una volta che i file sono stati creati, è necessario posizionarli nelle directory corrette:
\begin{itemize}
\item Il file di esecuzione nella directory src;
\item file di configurazione in src/conf; 
\item file di init in src/data.
\end{itemize}
Se i file sono nelle posizioni corrette si può avviare l'esecuzione. Per l'esecuzione su una sola macchina è necessario avviare almeno due peer seguendo questi semplici passi:
\begin{enumerate}
\item Posizionarsi in src;
\item Eseguire ./tortella ./conf/tortella.conf
\item Eseguire ./tortella ./conf/tortella1.conf ./data/init_data.  
\end{enumerate}
\section{Esecuzione}
L'esecuzione qui presentata viene eseguita su macchina locale simulando la presenza di tre peer nella rete. I tre peer, che per comodità verranno chiamati simo, lore, ibra saranno inizialmente connessi nel seguente modo: simo non conosce alcun peer vicino, lore e ibra invece conosceranno come unico vicino simo. I tre peer avviano l'applicazione e simo crea una chat dal nome "prova":
\begin{figure}[H]
\begin{center}
\includegraphics[scale=0.5]{etc/crea_chat.png}
\caption{Creazione della chat}
\label{crea\_chat}
\end{center}
\end{figure}
\begin{figure}[H]
\begin{center}
\includegraphics[scale=0.5]{etc/apertura_chat.png}
\caption{Apertura della chat}
\label{apertura\_chat}
\end{center}
\end{figure}
Lore cerca la chat "prova" e si connette:
\begin{figure}[H]
\begin{center}
\includegraphics[scale=0.5]{etc/ricerca_lore.png}
\caption{Ricerca della chat}
\label{ricerca\_lore}
\end{center}
\end{figure}
\begin{figure}[H]
\begin{center}
\includegraphics[scale=0.5]{etc/join_lore.png}
\caption{Join di lore}
\label{join\_lore}
\end{center}
\end{figure}
ibra crea una chat di nome "pippo"
\begin{figure}[H]
\begin{center}
\includegraphics[scale=0.5]{etc/crea_chat2.png}
\caption{Creazione di una nuova chat}
\label{crea\_chat2}
\end{center}
\end{figure}
Nell'attesa che qualcuno si connetta alla chat appena creata, ibra cerca la chat "prova" e si connette
\begin{figure}[H]
\begin{center}
\includegraphics[scale=0.5]{etc/ricerca_chat_2.png}
\caption{Ricerca della chat prova}
\label{ricerca\_chat\_2}
\end{center}
\end{figure}
\begin{figure}[H]
\begin{center}
\includegraphics[scale=0.5]{etc/join2.png}
\caption{Join di ibra}
\label{join2}
\end{center}
\end{figure}
I tre utenti comunicano tra di loro
\begin{figure}[H]
\begin{center}
\includegraphics[scale=0.5]{etc/conversazione_chat.png}
\caption{Conversazione tra gli utenti}
\label{conversazione\_chat}
\end{center}
\end{figure}
lore invia un messaggio solo a simo (in realtà avrebbe potuto mandarlo a più utenti, ma essendo la chat composta da soli 3 utenti non avrebbe avuto senso mandare il messaggio a tutti)
\begin{figure}[H]
\begin{center}
\includegraphics[scale=0.5]{etc/invio_sottolista.png}
\caption{Invio di un messaggio ad un sottoinsieme degli utenti}
\label{invio\_sottolista}
\end{center}
\end{figure}
lore invia un messaggio privato a ibra, che viene ricevuto correttamente da quest'ultimo
\begin{figure}[H]
\begin{center}
\includegraphics[scale=0.5]{etc/pm_ibra.png}
\caption{Invio di un messaggio privato}
\label{pm\_ibra}
\end{center}
\end{figure}
\begin{figure}[H]
\begin{center}
\includegraphics[scale=0.5]{etc/ricezione\_pm.png}
\caption{Ricezione di un messaggio privato}
\label{ricezione\_pm}
\end{center}
\end{figure}
infine lore effettua il leave dalla chat e in seguito simo chiude l'applicazione inviando un bye ad ibra
\begin{figure}[H]
\begin{center}
\includegraphics[scale=0.5]{etc/leave_lore.png}
\caption{Leave dalla chat prova}
\label{leave\_lore}
\end{center}
\end{figure}
\begin{figure}[H]
\begin{center}
\includegraphics[scale=0.5]{etc/bye_simo.png}
\caption{Ricezione di un bye}
\label{bye\_simo}
\end{center}
\end{figure}
